\documentclass[../manual.tex]{subfiles}
\input{arduinoLanguage.tex}

\begin{document}

\expname{Control speed of stepper motor using joystick}


\expaim
To control speed of stepper motor using a joystick

\expdesc
This experiment demonstrates controlling the stepper motor speed based on the position of the joystick

\expcircuitconnection
\begin{itemize}
	\item Using the 8-pin RMC connector, connect one end to the Arduino Stepper port and the other end to the port near the Stepper Motor
	\item Using the 8-pin RMC connector, connect one end to the Arduino Joystick port and the other end to the port near the Joystick
\end{itemize}

\expprogram
\lstinputlisting[language=Arduino]{Code/Ex15_StepperJoystick.cpp}


\expexplanation
To use the Stepper motor, we make use of the library \textless \textbf{AccelStepper.h}\textgreater. The line \textbf{stepper.setSpeed(0)} and \textbf{stepper.runSpeed()} sets the speed of the Stepper Motor.
This program shows how we can take input from one device and control the output of another device based on if, else if, else conditions.

\expprocedure

\begin{enumerate}
	\item Make the circuit connections as mentioned above.
	\item Install library \textbf{Stepper.h}, if not installed already
	\item Upload the above program to the Arduino Nano.
	\item Move the joystick around and observe the stepper motor
\end{enumerate}

\textbf{Follow steps 3-9 Getting Started section, to upload the code.
	Instructions to install libraries are also given in the Getting Started section} \\

\expresult
On changing the position of the joystick, the speed of the stepper motor should change.

\end{document}